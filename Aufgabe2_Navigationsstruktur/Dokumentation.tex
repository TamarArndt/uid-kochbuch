\documentclass[parskip,10pt,abstracton]{scrartcl}
\usepackage[top=3cm, bottom=3cm, left=3cm, right=3cm]{geometry}
\usepackage{polyglossia}
\setmainlanguage{german}
\pagenumbering{gobble}

\usepackage{setspace}
\onehalfspacing

% ------------------------------------------------------------------------------------
% packages
\usepackage{graphicx}
\usepackage{tikz}
\usepackage{enumerate}
\usepackage{xcolor}
\definecolor{inhalt}{HTML}{A1D490}
\definecolor{rahmen}{HTML}{ABBEC4}


% ------------------------------------------------------------------------------------
% Header
% ------------------------------------------------------------------------------------
\renewcommand*{\maketitle}{%
	{\centering\LARGE\sffamily\bfseries Aufgabe 2: Entwicklung einer Navigationsstruktur \par}
	\vspace{3em}
}

% ====================================================================================
% Document
% ====================================================================================
\begin{document}

\maketitle

% ------------------------------------------------------------------------------------

\textbf{kurze Erklärung:}\\
Der erste Navigationspunkt unserer Webseite ist Home/Startseite, von der ein Zugriff auf alles weitere möglich ist.

Unsere Webseite soll folgende Inhalte beinhalten:\\
Verschiedene Rezepte geordnet nach vegan, vegetarisch, Fleisch- und Fischgerichte, Gemüsegerichte und Länderspäzifische Gerichte. \\
Des Weiteren sind Backrezepte wie Plätzchen, Kuchen, Auflauf, aber auch andere süße oder herzhafte Rezepte zu finden. \\
Außerdem werden es vermutlich Rezepte für Soßen, Dips, Knabberzeug, Salate und Teiggerichte auf der Webseite zu finden sind. 

Deshalb sind die angehenden Stichpunkte auch als Kategorien zu finden. 

Des Weiteren wird es verschiedene Techniken zur Zubereitung von Gerichten geben. 
Zudem wird es einen persönlichen Bereich mit einer Login-Funktion geben, in der Rezepte gespeichert und verwaltet werden können.



\textbf{Inhalte in Oberbegriffen zusammengefasst:} \\
Home | Kochen | Backen | meine Rezepte

Der Bereich 'Techniken' ist in alle Rezept-Unterseiten integriert, also kein eigenständiger Navigationsbereich.
\newpage
\textbf{Schema zur Organisation der Begriffe:}\\
Hierarchische Struktur



\tikzset{
  basic/.style  = {draw, rounded corners=2pt, text width=2cm, font=\sffamily, rectangle, text height = 1em, align=center},
  root/.style   = {basic, thin, align=center},
  level 2/.style = {basic, thin,align=center,text width=8em},
  level 3/.style = {basic, thin, align=left, text width=6.5em}
}

\begin{tikzpicture}[
  level 1/.style={sibling distance=40mm},
  edge from parent/.style={-,draw},
  >=latex]

\node[root] {Kochbuch}
  child {node[level 2,fill=rahmen] (c1) {Home}}
  child {node[level 2,fill=inhalt] (c2) {Kochen}}
  child {node[level 2,fill=inhalt] (c3) {Backen}}
  child {node[level 2,fill=rahmen] (c4) {persönl. Bereich}};

\begin{scope}[every node/.style={level 3}]

\node [below of = c2, xshift=15pt,fill=inhalt] (c21) {vegan};
\node [below of = c21,fill=inhalt] (c22) {italienisch};
\node [below of = c22,fill=inhalt] (c23) {Suppe};
\node [below of = c23,fill=inhalt] (c24) {$\cdots$};

\node [below of = c3, xshift=15pt,fill=inhalt] (c31) {Kuchen};
\node [below of = c31,fill=inhalt] (c32) {Kleingebäck};
\node [below of = c32,fill=inhalt] (c33) {$\cdots$};

\node [below of = c4, xshift=15pt,fill=rahmen] (c41) {Login/Logout};
\node [below of = c41,fill=rahmen] (c42) {eigene Rezepte};
\node [below of = c42,fill=rahmen] (c43) {Einstellungen};
\end{scope}

% lines
\foreach \value in {1,...,4}
  \draw[-] (c2.195) |- (c2\value.west);
\foreach \value in {1,...,3}
  \draw[-] (c3.195) |- (c3\value.west);
\foreach \value in {1,...,3}
  \draw[-] (c4.195) |- (c4\value.west);

\end{tikzpicture}

\colorbox{inhalt}{Inhaltlich/Rezepte}

\colorbox{rahmen}{Rahmen / äußere Struktur / organisatorisches}

Für die Webseite eignet sich eine hierarchische Struktur.
Die Hauptbereiche sind Home, Kochen, Backen und der persönliche Bereich. 

Home ist die Startseite.\\
Kochen und Backen sind jeweils in Unterkategorien aufgedröselt, die zu den jeweiligen Rezepten führen. \\
Der persönliche Bereich ist einerseits die Login-Funktion und andererseits der Bereich für eigene Rezepte.
%Parallel zu dieser Struktur gibt es den persönlichen
%Der persönliche Bereich ist aufgeteilt in den Login und


\end{document}



