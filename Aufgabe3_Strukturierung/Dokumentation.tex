\documentclass[parskip,10pt,abstracton]{scrartcl}
\usepackage[top=3cm, bottom=3cm, left=3cm, right=3cm]{geometry}
\usepackage{polyglossia}
\setmainlanguage{german}
\pagenumbering{gobble}

\usepackage{setspace}
\onehalfspacing

% ------------------------------------------------------------------------------------
% packages
\usepackage{graphicx}
\usepackage{tikz}
\usetikzlibrary{arrows,shapes,positioning, shadows,trees}
\usepackage{enumerate}

% ------------------------------------------------------------------------------------
% Header
% ------------------------------------------------------------------------------------
\renewcommand*{\maketitle}{%
	{\centering\LARGE\sffamily\bfseries Aufgabe 3: Entwürfe \par}
	\vspace{3em}
}

% ====================================================================================
% Document
% ====================================================================================
\begin{document}

\maketitle

% ------------------------------------------------------------------------------------

Chris 1:

Vorteile: \\
- einfach und übersichtlich\\
- Seite nicht überladen

Nachteile: \\
- Menüpunkte Backen/Kochen können unübersichtlich werden bei zu vielen Unterkategorien\\
- unklar, was Login beinhaltet (welche Funktionen hat man davon?) -> Hinweis auf persönliche Funktionen / Möglichkeiten

Chris2:

Vorteile:\\
- klassischer Aufbau und daher intuitives Zurechtfinden möglich (Position von Login und Suche) \\
- übersichtliches Menü, das sich anpasst (kann klein und groß sein, je nach Bedarf)
-

Nachteile:\\
- Seitenmenü kann problematisch werden, wenn Menüpunkte zu lang sind.


Velat 1:

Vorteile:\\
- es ist ersichtlich, dass es einen eigenen Kochbuchbereich gibt, für den man sich einloggen muss.
- innovativer Suchbereich
- Suchbereich als eigener Frame jederzeit sichtbar


Nachteile:\\
- Menüpunkte haben zu viele Unterpunkte im Dropdownmenü
- Suchbereich verbraucht Platz von Inhaltsseite
- Suchfunktion zu kompliziert

Velat 2:

Vorteile:\\
- seitlicher Suchbereich: Kategorien können aus- und eingeklappt werden: übersichtlich
- mehr Platz für Inhalt da Suchbereich verbergbar ist

Nachteile:\\
- im Menü sowohl mein Kochbuch als auch Login -> Unterschied?

Tamar 1:

Vorteile: \\
- klassischer Aufbau
- Suche jederzeit sichbar
- übersichtlich

Nachteile:
- Rezept hinzufügen sollte unter mein Kochbuch kommen, da es sonst nicht in die Navigationsstruktur passt
- Suchfunktion verbraucht Platz
- auf Rezeptseite muss Rezept mehr Platz einnehmen

Tamar 2:

Vorteile: \\
- mehr Platz
- aufgeräumter

Nachteile:
- Suche als eigene Seite hat Nachteil, dass man ausgewählte Kategorien nicht nebenher sieht.
- Suchseite zu überladen




Elemente, die beibehalten werden sollen:

- seitlich Rezeptnavigation: enthält Hauptrezeptnavigation (Kochen / Backen), sowie Suche und Kategorien
  einzelne Menüpunkte ausklappbar
  seitliche Rezeptnavigation kann versteckt / einklappt werden
- oben rechts Login bzw. eigenes Kochbuch
- 


\end{document}



