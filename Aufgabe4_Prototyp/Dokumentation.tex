\documentclass[parskip,10pt,abstracton]{scrartcl}
\usepackage[top=3cm, bottom=3cm, left=3cm, right=3cm]{geometry}
\usepackage{polyglossia}
\setmainlanguage{german}
\pagenumbering{gobble}

\usepackage{setspace}
\onehalfspacing

% ------------------------------------------------------------------------------------
% packages
\usepackage{graphicx}
\usepackage{tikz}
\usetikzlibrary{arrows,shapes,positioning, shadows,trees}
\usepackage{enumerate}

% ------------------------------------------------------------------------------------
% Header
% ------------------------------------------------------------------------------------
\renewcommand*{\maketitle}{%
	{\centering\LARGE\sffamily\bfseries Aufgabe 4: Der Papierprototyp \par}
	\vspace{3em}
}

% ====================================================================================
% Document
% ====================================================================================
\begin{document}

\maketitle

% ------------------------------------------------------------------------------------

% AUFGABE
  % Prototyp mittels Fotos und Erklärungen dokumentieren:
  % Grundaufbau
  % Funktionalitäten
  % 3 Szenarien durchgehen

% ------------------------------------------------------------------------------------
\section*{Grundaufbau}

Startseite:
\begin{center}
\includegraphics[scale=0.4]{Prototyp/home.png}
\end{center}

Die Startseite enthält einen kurzen Begrüßungstext mit einer kurzen Erklärung der Navigation und Bilder von verschiedenen Rezepten.\\
Rechts oben befindet sich der Menüpunkt der zum persönlichen Kochbuch bzw. Login führt.\\
Auf der linken Seite gibt es die Möglichkeit, die Rezepte nach Belieben zu filtern. Entweder nach Suchbegriffen oder nach verschiedenen Kategorien, die auf dem Prototypen in einer Auswahl dargestellt sind.
\newpage
Rezeptseite:
\begin{center}
\includegraphics[scale=0.4]{Prototyp/plätzchenrezeptseite.png}
\end{center}

Eine Rezeptseite enthält ein Bild, Zutaten, Zubereitung und Kommentare zu einem Rezept. In dem Bereich auf der rechten Seite befindet sich der Button, um das Rezept zu speichern. Ist man nicht eingeloggt, wird man aufgefordert, sich einzuloggen. 
Außerdem befinden sich auf der Seite Informationen zum Aufwand des Gerichtes und die Kategorie, in der es sich befindet. 
\newpage
% ------------------------------------------------------------------------------------
\section*{Funktionalität der Navigation}

Einklappen der Seitennavigation:
\begin{center}
\includegraphics[scale=0.4]{Prototyp/home_menuhidden.png}
\end{center}

Die Seitennavigation kann durch Betätigung des Pfeiles auf der linken Seite ein- bzw. ausgeklappt werden. 

horizontales Menü: meine Rezepte
\begin{center}
\includegraphics[scale=0.4]{Prototyp/menu_eigeneRezepte_menuhidden.png}
\end{center}


Die einzelnen Menüpunkte sind durch Anklicken des Pfeiles aufklappbar:

Menü: Kochen
\begin{center}
\includegraphics[scale=0.4]{Prototyp/menu_kochen.png}
\end{center}

Menü: Backen
\begin{center}
\includegraphics[scale=0.4]{Prototyp/menu_backen.png}
\end{center}
\newpage
Menü: Zutat
\begin{center}
\includegraphics[scale=0.4]{Prototyp/menu_zutat.png}
\end{center}


\pagebreak
% ------------------------------------------------------------------------------------
\section*{Szenarien}

\subsection*{Szenario 1: Max will Plätzchen backen}

Er verwendet die Suchfunktion, gibt "Plätzchen" ein und erhält Plätzchenrezepte.
\begin{center}
\includegraphics[scale=0.4]{Prototyp/plätzchensuche.png}
\end{center}

Er wählt ein Rezept aus und wird auf die entsprechende Rezeptseite gebracht.
\begin{center}
\includegraphics[scale=0.4]{Prototyp/plätzchenrezeptseite.png}
\end{center}


\subsection*{Szenario 2: Karl hat nur bestimmte Zutaten da}

Er wählt im Seitenmenü "Zutat" aus...
\begin{center}
\includegraphics[scale=0.4]{Prototyp/menu_zutat.png}
\end{center}

... und gibt die Zutaten ein, die ihm zur Verfügung stehen: Nudeln, Hackfleisch und Tomaten
\begin{center}
\includegraphics[scale=0.4]{Prototyp/menu_zutat_nudeln.png}\\[1em]
\includegraphics[scale=0.4]{Prototyp/menu_zutat_nudelnhackfleisch.png}\\[1em]
\includegraphics[scale=0.4]{Prototyp/menu_zutat_nudelnhackfleischtomate.png}
\end{center}
Er bekommt eine Auswahl von Rezepten, die zu seinen Zutaten passen.

\newpage
\subsection*{Szenario 3: Anna lässt sich inspirieren}

Anna klappt das Seitenmenü ein, um auf der gesamten Breite die Bilder der Rezepte betrachten zu können.
\begin{center}
\includegraphics[scale=0.4]{Prototyp/home_menuhidden.png}
\end{center}

Sie scrollt durch die Bilder und sieht sich ein bestimmtes Rezept an.
\begin{center}
\includegraphics[scale=0.4]{Prototyp/rezeptseite_menuhidden.png}
\end{center}

Sie klickt auf "Rezept merken" und sieht sich dann unter "meine Rezepte" ihre bisher gespeicherten Rezepte an.
\begin{center}
\includegraphics[scale=0.4]{Prototyp/menu_eigeneRezepte_menuhidden.png}\\[1em]
\includegraphics[scale=0.4]{Prototyp/meineRezepte_menuhidden.png}
\end{center}

\end{document}



