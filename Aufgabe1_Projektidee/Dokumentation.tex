\documentclass[parskip,10pt,abstracton]{scrartcl}
\usepackage[top=3cm, bottom=3cm, left=3cm, right=3cm]{geometry}
\usepackage{polyglossia}
\setmainlanguage{german}
\pagenumbering{gobble}

\usepackage{setspace}
\onehalfspacing

% ------------------------------------------------------------------------------------
% packages
\usepackage{graphicx}
\usepackage{tikz}
\usetikzlibrary{arrows,shapes,positioning, shadows,trees}
\usepackage{enumerate}

% ------------------------------------------------------------------------------------
% Header
% ------------------------------------------------------------------------------------
\renewcommand*{\maketitle}{%
	{\centering\LARGE\sffamily\bfseries Aufgabe 1: Projektbeschreibung \par}
	\vspace{3em}
}

% ====================================================================================
% Document
% ====================================================================================
\begin{document}

\maketitle

% ------------------------------------------------------------------------------------

\section*{a) Überblick}

Die Idee für das Projekt ist eine \textit{Kochbuch-Webseite} zu erstellen, die für die im folgenden beschriebene Zielgruppe das Zubereiten von Gerichten erleichtern soll.

\textbf{Inhalt}\\
Die Seite bietet eine Sammlung von Rezepten aller Art.
Die Rezepte sind in verschiedene Kategorien eingeordnet und können sortiert und durchsucht werden.
Nutzer können Rezepte bewerten, hinzufügen, kommentieren und ein eigenes digitales Kochbuch verwalten, in dem sie sich Rezepte speichern.
Innerhalb der Rezepte gibt es zu ausgewählten Techniken und Angaben Hinweise für unerfahrene Köche. Der Nutzer kann z.B. eine Erklärung dazu erhalten, was eine Prise Salz ist.

\textbf{Zielgruppe}\\
Zu der Zielgruppe gehören 16 bis 40 jährige Gelegenheitsköche, die der digitalen Version eines Kochbuchs offen gegenüberstehen.
Einerseits spricht die Seite unerfahrene Köche an, die sich über Rezepte und bestimmte Techniken informieren wollen. Andererseits ist die Seite geeignet für inspirationssuchende Gelegenheitsköche, die ein Rezept für einen bestimmten Anlass suchen. Erfahrene Köche haben die Möglichkeit, neue Rezepte zu finden, diese zu speichern und schnell darauf zuzugreifen. 

\textbf{Ziele} \\
Eine der beiden Hauptfunktionen der Webseite ist diese zu nutzen, um sich von den gegebenen Rezepten inspirieren zu lassen. Eine andere Hauptfunktion besteht darin, die Webseite nach bestimmten Rezepten zu durchsuchen. 

% Notizen:
%Features: \\
%- Rezepte merken (eigene Rezeptsammlung erstellen)  $\to$ login \\
%- Rezepte hinzufügen \\
%- kategorisieren / taggen $\to$ Schlagwörter vorschlagen \\
% - suchfunktion \\
% - Tipps (Hinweise zu Techniken) \\
% 
% Zielgruppe: \\
% alle die gerne Kochen \\
% leicht Technikaffin - keine Angst vor digitalen Kochbüchern \\
% 16 - 40 Jahre
% 
% Inspirations suchende \\
% Kochneulinge \\
% auch erfahrene, die ihre Rezepte speichern und schnell wieder finden wollen \\
% Gelegenheitsköche/bäcker, die nur zu bestimmten Anlass etwas brauchen -> Geburtstagskuchen


\pagebreak
\section*{b) Szenarien}

\begin{enumerate}[(1)]
 \item Max ist 16 Jahre alt (Schüler) und verbringt nicht viel Zeit in der Küche, möchte aber vor Weihnachten Plätzchen backen. Er verwendet die Suchfunktion, um schnell Rezepte für Plätzchen zu finden. Von einigen sieht er sich die Bilder und Kommentare an und entscheidet sich für ein einfaches Rezept.
 
 %Plätzchen backen, sonst kein Vielkocher $\to$ Suchfunktion oder über Kategoriennavigation \\ 16 jährige Schülerin
 
 \item Karl ist Student und hat Hunger. Er hat aber nur Nudeln, Hackfleisch und Tomaten zuhause. Auf der Webseite gibt er diese Zutaten ein und erhält eine kleine Auswahl von Rezepten. 
 
 %Kochen nach Zutaten die zur Verfügung stehen - Rezept finden $\to$ verwendet Suchfunktion nach Zutat \\ fauler Student (oder WG)
 
 \item Anna ist 40 und Hausfrau. Sie möchte am Sonntag für ihre Familie ein neues Rezept ausprobieren. Auf der Webseite klappt sie das Seitenmenü ein und scrollt durch die Bilder. Sie öffnet ein Rezept, das ihr gefällt. Danach speichert sie das Rezept ab und betrachtet ihre gespeicherten Rezepte. 
 
 %Inspirationssuchender $\to$ verwendet Merkfunktion \\ Hausfrau ca 40, will neues ausprobieren
\end{enumerate}


\end{document}



